\section{Introducción}
\subsection{Objetivo}
    Dada la actual situación sanitaria (marzo de 2020) y las preocupantes previsiones de futuro, interesa disponer de un modelo de respirador que permita su fabricación masiva con elementos fácilmente disponibles en el mercado y que permita ayudar a respirar a personas afectadas por el coronavirus COVID-19 sin excesivos requerimientos externos. Un enchufe debe ser suficiente para que funcione, sin necesitar de aire a presión ni otros requerimientos, aunque es posible el aporte de aire enriquecido con oxígeno en la toma de entrada.
\subsection{Requisitos}
    El respirador será “no invasivo”, aportará aire a presión al paciente por medio de una mascarilla. Los tiempos de inspiración serán regulables en duración e intensidad. Los ciclos de espiración serán pasivos, con presión positiva y regulables en duración.\\
    Se han escogido las especificaciones al efecto publicadas en el Reino Unido \cite{MHRA} por ser bastante completas y realistas, dadas las circunstancias.\\
    Deberá ser barato y rápido de construir, con funcionamiento sencillo y comprensible y componentes  económicos y fáciles de conseguir. Se emplearán materiales no contaminantes, ni en su operación ni en su destrucción.\\
    También deberá cumplir con los requisitos que exige la Agencia Española de Medicamentos y Productos
    Sanitarios. 