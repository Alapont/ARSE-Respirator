\section{Operación}
    Desde el panel frontal se seleccionará la potencia de la turbina (de 0 a 100\%), el tiempo de inspiración (desde mínimo a máximos programados) y el tiempo de espiración (con mínimo y máximo programados).\\
    El display ofrece la siguiente información:
    \begin{itemize}
        \item Fila Superior
        \begin{itemize}
            \item[--] P: potencia, de 0 a 99\%
            \item[--] R: respiraciones por minutos (RPM)
            \item[--] E/I fase actural I0, I1, E0, E1
            \item[--] i: tiempo de inspiración, segundos
        \end{itemize}
        \item Fila inferior
        \begin{itemize}
            \item[--] I:E: relación tiempo insp. / tiempo esp
            \item[--] E: Tiempo de espiración, segundos
        \end{itemize}
    \end{itemize}
    
    El programa lleva a cabo el ciclo inspiración+espiración en cuatro fases: I0, I1, E0 y E1 en las que las fases I0 y E0 son muy breves y solo se usan para ajustes técnicos (como la arrancada del motor, por ejemplo), la casi totalidad de la inspiración se lleva a cabo en I1. E0 no se usa actualmente\\
    
    Los tiempos seleccionados corresponden a;
    \begin{itemize}
        \item Tiempo de inspiración: I0 + I1
        \item Tiempo de espiración: E0 + E1
    \end{itemize}
    
    En el actual diseño la selección de tiempos se hace escogiendo los tiempos de inspiración y espiración y mostrando las RPM y la relación I:E. Puede fácilmente cambiarse el programa para que los potenciómetros seleccionen la relación I:E y las RPM.